\documentclass[10pt,a4paper]{article}
\usepackage[utf8]{inputenc}
\usepackage{amsmath}
\usepackage{amsfonts}
\usepackage{amssymb}
\usepackage{graphicx}
\usepackage[ngerman]{babel}
\usepackage[a4paper, portrait, margin=1.2in]{geometry}
\author{AW, TS, CS, AD, RB}
\title{Großes Studienprojekt}


\begin{document}
\maketitle
\begin{abstract}
  Da IPv4 aufgrund von immer weiter steigenden Nutzerzahlen
  und internetfähigen Geräte bald nicht mehr als
  Standardadressraum verwendet werden kann und IPv6 eine
  mögliche Lösung dieses Problems ist, haben wir uns innerhalb
  unseres großen Studienprojektes mit den Aufsetzen und
  Verwalten eines IPv4 bzw. IPv6 Netzwerkes
  auseinandergesetzt.
\end{abstract}
\section{Fragestellung}
Um sich mit dem Zusammenspiel von Geräten innerhalb eines
Netzwerkes vertraut zu machen, war gefordert das folgende
Diagramm als Netzwerk umzusetzen:
\begin{figure}[ht]
  % \includegraphics[width=\linewidth]{img/network_topography.png}
  \centering
  \caption{Aufbau des Netzwerks}
\end{figure}
\par
Die mit N1 bis N4 beschriftete Dreiecke bezeichnen die von uns
aufgestellte Subnetze die durch die Router (graue Ovale, R1 bis R4)
aufgebaut werden, und die blaue Vierecke ein Switch. Das \glqq
einfahrt Verboten\grqq~ Schild symbolisiert zusätzlich eine von uns
konfigurierte Firewall.
\par
Jeder Router sollte ein eigenen DHCP-Server für das eigene Netzwerk
bereitstellen. Alle Router erkennen sich gegenseitig über statische
Routes, wobei um zwischen R2 und R4 zu kommunizieren zwei Hops (über
R1) benötigt werden. Außerdem werden die Verbindungen zwischen R2 und
R3, und R3 und R4 über einem Switch und durch zwei unabhängige VLANs
realisiert, die über den Switch in der Mitte konfiguriert werden.
\par
Hinter R1 ist eine Verbindung zum Internet das durch die Firewall
geschützt wird. Die Firewall ist so konfiguriert das sie den
ausgehenden Verkehr nicht einschränkt aber nur dann eingehende Pakete
weiterleitet wenn sie im Zusammenhang mit einem ausgehenden Paket
stehen.
\par
Der innere Switch sollte zusätzlich so konfiguriert sein das er die
Verbindungen zu den Routern über die verbliebene Ports spiegelt.

\section{Herangehensweise}
Nach ersten Recherchen bezüglich der Router und deren
Funktionsweise fiel auf, dass diese auf sehr veralteten
Betriebssystemversionen liefen. Deshalb war einer der ersten
Arbeitsschritte das Aktualisieren der Firmware. Hiernach ist
aufgefallen das in der aktualisierten Version eine GUI zur
Verfügung gestellt wurde, mit deren Hilfe jegliche
Einstellungen konfiguriert werden konnten.
\par
Bevor die Aufgabe bearbeitet wurde war eine Einarbeitung in die
Funktionen und Eigenschaften des Systems erforderlich, um eine
vollständige Umsetzung der Fragestellung zu realisieren. Dazu wurde
ein simples Netzwerk mit Hilfe der Konsole konfiguriert. In diesem
Netzwerk wurden alle für die Aufgabenstellung wichtige
Komponenten(z.B. DHCP, Firewall, Static Routes, usw.) eingebunden, um
ein tieferes Verständnis dieser zu gewinnen.
\par
Nach ersten Konfigurationen wurde mit Hilfe der GUI geprüft, ob alle
Einstellungen erfolgreich übernommen wurden. Dabei ist aufgefallen das
mittels der GUI einige Konfigurationen deutlich einfacher und
übersichtlicher zu handhaben sind(Bspw. Static Routes). Daher wurde
zur Umsetzung der Aufgabenstellung eine Kombination aus Konsole und
GUI verwendet.
\par
Bei der Bearbeitung der Aufgabe musste zu aller erst das Netzwerk auf
physischer Ebene, wie auf Abbildung 1 zu sehen, zusammen gesetzt
werden. Dabei war eine übersichtliche Verkabelung wichtig, um die
Subnetze auch unproblematisch unterscheiden und visualisieren zu
können. Des weiteren wurden die Router etikettiert um weitere
Übersicht und Organisation zu erschaffen.
\par
Wichtig war die Dokumentation des Verlaufs der Umsetzung als auch eine
Möglichkeit ältere Konfigurationen zu sichern und wieder
herzustellen. Hierzu wurde Gitub verwendet, dies erlaubte eine bessere
Organisation des Projekts und effektivere Zusammenarbeit.


\section{Konfiguration}
	Router 0 dient als Schnittstelle zum Internet, deshalb ist er der einzige Router, der kein eigenes Sub-Netz aufspannt. Im Gegensatz zu den anderen Routern konnte an eth1 und eth2 (die Verbindung zu R1 und R2) keine festen IP-Adressen vergeben werden. Stattdessen läuft ein DHCP-Server, der dem jeweiligen Anschluss eine passende IP-Adresse vergibt. Das ist notwendig, da sonst kein Zugriff auf das Webinterface des Routers mehr möglich wäre. Die drei anderen Router sind über eine IP-Adresse aus dem Adressraum ihres jeweiligen Subnetzes erreichbar (z.B. Router 3: 192.168.3.1) \par
	
	Router 1-3 haben alle eine recht ähnliche Konfiguration: alle bieten auf eth0 ein eigenes Subnetz. Über eth1 und eth2 sind sie mit den anderen Routern verbunden. Die Router sowie die Geräte in den Subnetzen können sich über statische Routen erreichen. \par 
	
	Die Firewall wurde entsprechend der Vorgabe in der Aufgabenstellung konfiguriert. Die Konfiguration erfolgte hauptsächlich über die in der Router-Software verfügbare Shell. Diese ist wesentlich mächtiger als das integrierte Webinterface, insbesondere bei seltener verwendeten Features des Routers. \par 
	
	Das VLAN wurde durch die Konfiguration des entsprechenden Switches realisiert. Dies lässt sich mit der Konfigurationssoftware bewerkstelligen, die allerdings nur für Windows vorliegt (alternativ geschieht die Konfiguration über ein Web-Interface). Unsere Lösung sah zuerst vor,  die Ports 1-4 und 5-8 als zwei VLANs zu trennen. Da wir jedoch noch einen Port für das Monitoring benötigten, mussten wir das zweite VLAN um einen Port verkleinern. Somit kann nun der gesamte Traffic des Switches an Port 8 abgegriffen werden. \par 


\section{Netzwerkanalyse}Rudi


\section{Evaluation}Thomas
\end{document}
